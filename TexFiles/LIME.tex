\section{LIME}
\begin{frame}
	\frametitle{Requirements}
	\begin{Large}
		\textbf{What do we want:}
		\begin{itemize}
			\item Human Readable Model Explanation
			\item For Every Classifier
			\item For Every Input
		\end{itemize}
	~\newline
	\begin{center}
		\textbf{ features $\neq$ human readable }
	\end{center}
	~\newline
	\end{Large}
	To gain $readability$: 
	\begin{itemize}
		\item show influence relative to each other, not as numbers
		\item only show most important features
		\item use \textit{superpixels} instead of pixels
	\end{itemize}
\end{frame}

\begin{frame}
	\frametitle{Definitions}
	Let:
	\begin{enumerate}
		\item $G$ be any possible explanation model
		\item $g$ be our explanation Model
		\item $\Omega(g)$ the complexity of our Model
		\begin{itemize}
			\item Weights in a regressions model
			\item Depth of an decisiontree
			\item Number of trees in a random forest
		\end{itemize}
		\item $f: Features -> Class $ be the real classification
		\item $\Pi_x(z)$ as proximity-measure from $x$ to $z$
		\item $\mathcal{L}(f,g,\Pi_x)$ measure of un-faithfullness of $g$ compared to $f$ given the proxmity $\Pi_x$
	\end{enumerate}
\end{frame}

\begin{frame}
	\frametitle{Minimizing Fidelity $\cdot$ Interpretability}
	\begin{Large}
		Wanted: ~\newline
		\begin{center}
			$\xi(x) = argmin_{g\in G} ~ \mathcal{L}(f,g,\Pi_x) + \Omega(g)$
		\end{center}
		Read: 
		\begin{itemize}
			\item We want for every input $x$
			\item an explanation(-model)
			\item where complexity of $g$ and the failure of $g$ are minimal
			\item given a set of possible explanations $G$
		\end{itemize}
	\end{Large}
~\newline ~\newline 
We do so by picking samples $x^,$ as subsets from an input $x$ and optimizing our model $g$ 
\end{frame}

\begin{frame}
	\frametitle{Local Interpretable Model-Agnostic Explanations}
	\framesubtitle{The LIME-Algorithm}
	Here is the Pseudocode. 
\end{frame}
\begin{frame}
	\frametitle{Visualisation}
	Put the funky red-blue image with the red-crosses from the paper here
\end{frame}

\begin{frame}
	maybe: Example 
\end{frame}